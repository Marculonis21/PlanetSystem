\documentclass[a4paper, 12pt]{article}
\usepackage{titling}
\usepackage{titlesec}
\usepackage{amssymb}
\usepackage{pifont}


\usepackage{hyperref}
\hypersetup{
    colorlinks=true,
    linkcolor=black,
    % filecolor=magenta,
    urlcolor=cyan,
}
\urlstyle{same}

\newcommand{\cmark}{\ding{51}}
\newcommand{\xmark}{\ding{55}}

\renewcommand{\contentsname}{Obsah}
\renewcommand{\thesection}{\Roman{section}}

\titleformat{\section}
{\Large\bfseries}
{\Roman{section}.}
{0.5em}
{}


\titleformat{\subsection}
{\large\bfseries}
{\thesubsection.}
{0.5em}
{}

\title{
        \vspace{1in}
        \rule{\linewidth}{0.5pt}
		\usefont{OT1}{bch}{b}{n}
        \huge Uživatelská dokumentace \\PlanetSystem\\
        \vspace{-10pt}
        \rule{\linewidth}{1pt}
}
\author{
		\normalfont\normalsize
        Marek Bečvář\\[-3pt]\normalsize
        12.2.2021
}
\date{}


\begin{document}
\maketitle 
\newpage

\tableofcontents
\newpage

\section{O programu} 
\paragraph{}
PlanetSystem je programem pro Windows/Linux umožnující uživateli ve 2D vytvářet vlastní
simulované planetární systémy. Simulace pracují se skutečnými fyzikálními
závislostmi a vlastnostmi, které mohou být pro jednotlivá tělesa ve fázi
\\editování upravována. To, jak jednotlivé změny ovlivňují celý systém, může pak
uživatel sledovat v reálném čase v zobrazovacím okně.

Oblíbené simulace je pak možné ukládat a zpětně načítat s pomocí vlastního
speciálního menu. Program zároveň přichází s pár předem uloženými ukázkovými
simulacemi, demonstrující možnosti, kterých je možné v simulacích dosáhnout. 

\section{Instalace}
\paragraph{Python}
Projekt je vytvořen v programovacím jazyce Python verze 3.8.5. Pro maximální
funkčnost je doporučeno využívat tuto verzi, i když \\kompatibilita je
očekávána i s jinými verzemi Pythonu 3 (dokud je možná spolupráce s potřebnými
knihovnami).\\\\ Instalace možná z oficiálních stránek Python.org
\url{https://www.python.org/downloads/}.

\paragraph{Potřebné knihovny} Pro správnou funkčnost programu je potřeba mít k
základnímu Pythonu nainstalované ještě další knihovny. 


\begin{table}[h!]
\centering
\hspace*{-1.5cm}
\begin{tabular}{ |c|c|c| }
 \hline
 Jméno knihovny & Dokumentace & Standardní knihovna\\
 \hline
 \textbf{Enum} & \url{https://docs.python.org/3/library/enum.html} & \cmark\\
 \hline
 \textbf{Copy} & \url{https://docs.python.org/3/library/copy.html} & \cmark\\
 \hline
 \textbf{Os} & \url{https://docs.python.org/3/library/os.html} & \cmark\\
 \hline
 \textbf{Pickle} & \url{https://docs.python.org/3/library/pickle.html} & \cmark\\
 \hline
 \textbf{Random} & \url{https://docs.python.org/3/library/random.html} & \cmark\\
 \hline
 \textbf{Sys} & \url{https://docs.python.org/3/library/sys.html} & \cmark\\
 \hline
 \textbf{Numpy} & \url{https://numpy.org/doc/} & \xmark\\
 \hline
 \textbf{Pygame} & \url{https://www.pygame.org/docs/} & \xmark\\
 \hline
\end{tabular}

\vspace{0.25cm}
\footnotesize{Řada z těchto knihoven je považována za standardní (není
potřeba instalovat), ale pro úplnost jsou v tabulce výše uvedeny všechny.}
\end{table}

\pagebreak
Postup pro doinstalování potřebných knihoven je jednoduchý, přesněji popsaný v
těchto zdrojích:\\
\textbf{Numpy}: \url{https://numpy.org/install/}\\
\textbf{Pygame}: \url{https://www.pygame.org/wiki/GettingStarted}\\

Po nainstalování potřebných knihoven je již program plně funkční a spustitelný
souborem \textbf{PlanetSystem.py}.


\end{document}
